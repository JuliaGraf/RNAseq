%%%%%%%% ICML 2023 EXAMPLE LATEX SUBMISSION FILE %%%%%%%%%%%%%%%%%

\documentclass{article}

% Recommended, but optional, packages for figures and better typesetting:
\usepackage{microtype}
\usepackage{graphicx}
\usepackage{subfigure}
\usepackage{booktabs} % for professional tables
\usepackage{animate}


\usepackage{tikz}
% Corporate Design of the University of Tübingen
% Primary Colors
\definecolor{TUred}{RGB}{165,30,55}
\definecolor{TUgold}{RGB}{180,160,105}
\definecolor{TUdark}{RGB}{50,65,75}
\definecolor{TUgray}{RGB}{175,179,183}

% Secondary Colors
\definecolor{TUdarkblue}{RGB}{65,90,140}
\definecolor{TUblue}{RGB}{0,105,170}
\definecolor{TUlightblue}{RGB}{80,170,200}
\definecolor{TUlightgreen}{RGB}{130,185,160}
\definecolor{TUgreen}{RGB}{125,165,75}
\definecolor{TUdarkgreen}{RGB}{50,110,30}
\definecolor{TUocre}{RGB}{200,80,60}
\definecolor{TUviolet}{RGB}{175,110,150}
\definecolor{TUmauve}{RGB}{180,160,150}
\definecolor{TUbeige}{RGB}{215,180,105}
\definecolor{TUorange}{RGB}{210,150,0}
\definecolor{TUbrown}{RGB}{145,105,70}

% hyperref makes hyperlinks in the resulting PDF.
% If your build breaks (sometimes temporarily if a hyperlink spans a page)
% please comment out the following usepackage line and replace
% \usepackage{icml2023} with \usepackage[nohyperref]{icml2023} above.
\usepackage{hyperref}


% Attempt to make hyperref and algorithmic work together better:
\newcommand{\theHalgorithm}{\arabic{algorithm}}

\usepackage[accepted]{icml2023}

% For theorems and such
\usepackage{amsmath}
\usepackage{amssymb}
\usepackage{mathtools}
\usepackage{amsthm}

% if you use cleveref..
\usepackage[capitalize,noabbrev]{cleveref}

%%%%%%%%%%%%%%%%%%%%%%%%%%%%%%%%
% THEOREMS
%%%%%%%%%%%%%%%%%%%%%%%%%%%%%%%%
\theoremstyle{plain}
\newtheorem{theorem}{Theorem}[section]
\newtheorem{proposition}[theorem]{Proposition}
\newtheorem{lemma}[theorem]{Lemma}
\newtheorem{corollary}[theorem]{Corollary}
\theoremstyle{definition}
\newtheorem{definition}[theorem]{Definition}
\newtheorem{assumption}[theorem]{Assumption}
\theoremstyle{remark}
\newtheorem{remark}[theorem]{Remark}

% Todonotes is useful during development; simply uncomment the next line
%    and comment out the line below the next line to turn off comments
%\usepackage[disable,textsize=tiny]{todonotes}
\usepackage[textsize=tiny]{todonotes}


% The \icmltitle you define below is probably too long as a header.
% Therefore, a short form for the running title is supplied here:
\icmltitlerunning{nextflow RNAseq pipeline}

\begin{document}

\onecolumn
\icmltitle{nextflow RNAseq pipeline}

% It is OKAY to include author information, even for blind
% submissions: the style file will automatically remove it for you
% unless you've provided the [accepted] option to the icml2023
% package.

% List of affiliations: The first argument should be a (short)
% identifier you will use later to specify author affiliations
% Academic affiliations should list Department, University, City, Region, Country
% Industry affiliations should list Company, City, Region, Country

% You can specify symbols, otherwise they are numbered in order.
% Ideally, you should not use this facility. Affiliations will be numbered
% in order of appearance and this is the preferred way.
\icmlsetsymbol{equal}{*}

\begin{icmlauthorlist}
\icmlauthor{Jana Hoffmann}{equal,first}
\icmlauthor{Julia Graf}{equal,second}
\icmlauthor{Jessie Midgley}{equal,third}
\end{icmlauthorlist}

% fill in your matrikelnummer, email address, degree, for each group member
\icmlaffiliation{first}{Matrikelnummer 5760486, jana2.hoffmann@student.uni-tuebingen.de, MSc Bioinformatics}
\icmlaffiliation{second}{Matrikelnummer 5656882, jul.graf@student.uni-tuebingen.de, MSc Bioinformatics}
\icmlaffiliation{third}{Matrikelnummer 6620875, jessie.midgley@student.uni-tuebingen.de, MSc Bioinformatics}

% You may provide any keywords that you
% find helpful for describing your paper; these are used to populate
% the "keywords" metadata in the PDF but will not be shown in the document
\icmlkeywords{Machine Learning, ICML}

\vskip 0.3in

% this must go after the closing bracket ] following \twocolumn[ ...

% This command actually creates the footnote in the first column
% listing the affiliations and the copyright notice.
% The command takes one argument, which is text to display at the start of the footnote.
% The \icmlEqualContribution command is standard text for equal contribution.
% Remove it (just {}) if you do not need this facility.

%\printAffiliationsAndNotice{}  % leave blank if no need to mention equal contribution
\printAffiliationsAndNotice{\icmlEqualContribution} % otherwise use the standard text.

\begin{abstract}
TODO add our abstract
\end{abstract}

% This is the template for a figure from the original ICML submission pack. In lecture 10 we will discuss plotting in detail.
% Refer to this lecture on how to include figures in this text.
%
% \begin{figure}[ht]
% \vskip 0.2in
% \begin{center}
% \centerline{\includegraphics[width=\columnwidth]{icml_numpapers}}
% \caption{Historical locations and number of accepted papers for International
% Machine Learning Conferences (ICML 1993 -- ICML 2008) and International
% Workshops on Machine Learning (ML 1988 -- ML 1992). At the time this figure was
% produced, the number of accepted papers for ICML 2008 was unknown and instead
% estimated.}
% \label{icml-historical}
% \end{center}
% \vskip -0.2in
% \end{figure}

\section{Introduction}
(a) Pipelines
(b) FAIR principles and reproducibility
(c) nf-core
(d) RNA-seq
RNA-sequencing is a technique used to quantify gene expression, and allows for comprehensive profiling of the transcriptome \cite{Kukurba2015}. One of the main goals of gene expression experiments is to identify transcripts that show differential expression under various conditions. A usual RNA-seq workflow involves isolating RNA from the sample, converting it to complementary DNA (cDNA), and sequencing it using an NGS platform. The generated FastQ files contain the sequencing reads of the sample, which can then be aligned to a reference genome using a mapping tool, in order to quantify the expression of genes. The quantification is usally inferred by analyzing the accumulation of read alignments mapped to the reference genome.

\section{Methods}
(a) How is an nf-core pipeline structured, explain the most important files
(b) where / how did you compute, which packages etc did you use (be FAIR when describing)
(c) ...

\section{Results}
(a) describe all aspects of your pipeline
(b) potentially describe aspects like speed with respect to parallelization (potentially compare
for a couple of files to running plain bash...)

\section{Discussion}
(a) How does nof-core achieve reproducibility and FAIR pipelines with focus on what your
pipeline would still need to be able to be FAIR, reprod.
(b) what is missing to your pipeline to be a strong RNA-seq workflow. what would be the next
steps - Outlook
While the pipeline successfully generates transcript quantification data, there are opportunities to improve its performance by incorporating additional processes. For instance, integrating support for Unique Molecular Identifiers (UMIs) would allow for their extraction and downstream read deduplication, improving the accuracy of the results and reducing potential biases. The strandedness of sequencing reads could be inferred automatically by combining the subsampling of FastQ files with pseudoalignment, to determine whether reads correspond to the original RNA sequence or its complementary cDNA. Stranded RNA-Seq data offers advantages over non-stranded data, by improving the accuracy of transcript assembly and differential expression \cite{Signal2022}. Additionally, the removal of rRNA from reads can lead to the better detection of mRNA transcripts, which is important for the analysis of differential expression \cite{Pastor2022}. Finally, the generation of additional bigWig coverage files would allow for clearer visualisation of transcript coverage across the genome.

\section*{Data Availability}
The source code for this paper is available on \href{{https://github.com/JuliaGraf/RNAseq}}{GitHub}.

\bibliography{bibliography}
\bibliographystyle{icml2023}

\end{document}


% This document was modified from the file originally made available by
% Pat Langley and Andrea Danyluk for ICML-2K. This version was created
% by Iain Murray in 2018, and modified by Alexandre Bouchard in
% 2019 and 2021 and by Csaba Szepesvari, Gang Niu and Sivan Sabato in 2022.
% Modified again in 2023 by Sivan Sabato and Jonathan Scarlett.
% Previous contributors include Dan Roy, Lise Getoor and Tobias
% Scheffer, which was slightly modified from the 2010 version by
% Thorsten Joachims & Johannes Fuernkranz, slightly modified from the
% 2009 version by Kiri Wagstaff and Sam Roweis's 2008 version, which is
% slightly modified from Prasad Tadepalli's 2007 version which is a
% lightly changed version of the previous year's version by Andrew
% Moore, which was in turn edited from those of Kristian Kersting and
% Codrina Lauth. Alex Smola contributed to the algorithmic style files.
